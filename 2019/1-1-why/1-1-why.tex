% 1-1-why.tex

%%%%%%%%%%%%%%%%%%%%
\documentclass[a4paper, justified]{tufte-handout}

\input{hw-preamble} % feel free to modify this file
%%%%%%%%%%%%%%%%%%%%
\title{第1讲: 为什么计算机能解题?}
\me{魏恒峰}{hfwei@nju.edu.cn}{}{}
\date{\zhtoday} % or like 2019年9月13日
%%%%%%%%%%%%%%%%%%%%
\begin{document}
\maketitle
%%%%%%%%%%%%%%%%%%%%
\noplagiarism % always keep this line
%%%%%%%%%%%%%%%%%%%%
\begin{abstract}
  \mfig{width = 0.60\linewidth}{figs/recursion-mirror}
  \begin{center}{\fcolorbox{blue}{yellow!60}{\parbox{0.50\textwidth}{\large 
    \begin{itemize}
      \item 体会``思维的乐趣''
      \item 初步了解递归与数学归纳法 
      \item 初步接触算法概念与问题下界概念
    \end{itemize}}}}
  \end{center}
\end{abstract}
%%%%%%%%%%%%%%%%%%%%
\beginrequired

% divide.tex

%%%%%%%%%%%%%%%
\begin{problem}[UD Problem $1.9$]
  Let $n$ be an odd integer. Prove that $n^3 − n$ is divisible by 24.
\end{problem}

\begin{solution}
\end{solution}
%%%%%%%%%%%%%%%

% lighter-coin.tex

%%%%%%%%%%%%%%%
\begin{problem}[$n$ 枚硬币]
  \mfig{width = 0.70\linewidth}{figs/scale}

  你有 $n$ 枚外观一模一样的硬币。
  已知其中有一枚假币,并且假币的质量比真币轻。\\
  现有一个带两个托盘的天平秤。
  请设计\red{``称量''}~\footnote{只允许使用``称量''操作。这是我们在做算法分析时关注的关键操作。}\red{方案}~\footnote{这就是算法。},找到这枚假币。

  请用尽可能简洁的自然语言或者伪代码描述你的称量方案。
  不要提交可执行代码。
\end{problem}

\begin{solution}
\end{solution}
%%%%%%%%%%%%%%%


% lighter-coin-lower-bound.tex

%%%%%%%%%%%%%%%
\begin{problem}[$n$ 枚硬币问题的下界]
  接上一题,
  \red{最少}
  ~\footnote{这就是问题的下界。显然,只考虑特定的\blue{\it 算法}是不够的;你要考虑\blue{\it 问题}本身的性质以及``称量''操作的本质。}
  需要称量多少次,才能找到这枚假币?
  请证明你的结论。
\end{problem}

\begin{solution}
\end{solution}
%%%%%%%%%%%%%%%

% 12coins.tex

%%%%%%%%%%%%%%%
\begin{problem}[$12$ 枚硬币 (UD Problem $1.8$)]
  你有 $12$ 枚外观一模一样的硬币。
  已知其中有一枚假币,其质量与真币不同。\\
  \red{但是,你不知道假币比真币轻还是重}。
  只称量三次,如何找出这枚假币,并确定它相对于真币的轻重?

  \mfig{width = 0.80\textwidth}{figs/try}
\end{problem}

\begin{solution}%
\end{solution}
%%%%%%%%%%%%%%%

%%%%%%%%%%%%%%%%%%%%
\beginoptional

% labelled-coin.tex

%%%%%%%%%%%%%%%
\begin{problem}[$n$ 枚硬币]
  你有 $n$ 枚外观一模一样的硬币。
  已知其中有一枚假币,其质量与真币不同。\\
  但是,你不知道假币比真币轻还是重。
  \red{好在,每个硬币都有一个标签 ``Possibly Heavy''}~\footnote{表示该硬币是真币或者比真币重。} \red{或者 ``Possibly Light''。}
  请设计``称量''方案,找出这枚假币,并确定它相对于真币的轻重。
\end{problem}

\begin{solution}
\end{solution}
%%%%%%%%%%%%%%%

%%%%%%%%%%%%%%%%%%%%
\end{document}